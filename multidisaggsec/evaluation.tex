\section{Evaluation}
%The evaluation tools has been discussed in prior work \cite{liang2010load}.
%
%To evaluate whether each devices is disaggregated or not,
%we need to evaluate two parts, one is how much energy
%is evaluated compared to the ground truth. Another is
%among those disaggregated energy, how much percentage
%are disaggregated right.
%
We use precision, recall and F-measure in our evaluation. The standard
definition of these metrics are:
%\begin{equation}
$\textrm{precision}=\frac{TP}{TP+FP}$, 
%\end{equation}
%\begin{equation}
$\textrm{recall}=\frac{TP}{TP+FN}$,
%\end{equation}
%\begin{equation}
$\textrm{F-measure}=\frac{1}{\frac{1}{\textrm{precision}}+\frac{1}{\textrm{recall}}}$
%\end{equation}

We need to define the notions of true/false positives
and negatives in the context of disaggregation.

Now suppose there is a ground truth time series $X$ with length T.
Denote the corresponding disaggregated time series by $X^*$.
For any time $t \in (0, T)$, there are two values: the
ground truth value $X_i(t)$ and the disaggregated value
$X_i^*(t)$. We define a parameter $\rho$ for the range of
true values $X_i(t)$ and another parameter $\theta$
as the noise.
For any given measurement, 
there are four total power values or water usage values at
each point: true positive $\Psi_{TPi}$,  false negative $\Psi_{FNi}$,
true negative $\Psi_{TNi}$, and false positive $\Psi_{FPi}$.

\noindent
1. When $X_i(t) > \theta$ and  $X_i^*(t)> \theta  $,
at this point the disaggregation is a true positive.
There are three situations in turn:

\noindent
1.1. When $  X_i(t) \times (1-\rho) <  X_i^*(t) <  X_i(t) \times (1+\rho)  $, then
\begin{eqnarray*}
 {\Psi}_{TPi} &=& X_i^*(t) \\
 \Psi_{FNi}&=&\Psi_{FPi} =\Psi_{TNi}=0
\end{eqnarray*}

\noindent
1.2. When $ X_i^*(t) < X_i(t) \times (1-\rho)$ , then only
the disaggregated power or water usage is considered as true positive and
the power or water usage that is not disaggregated is regarded as a false negative:
\begin{eqnarray*}
\Psi_{TPi}&=&X_i^*(t) \\
\Psi_{FNi}&=&X_i(t) - X_i^*(t) \\
\Psi_{FPi}&=&\Psi_{TNi}=0
\end{eqnarray*}
1.3 When $ X_i^*(t)>  X_i(t) \times (1+\rho) $, then
the disaggregated power or water usage is a true positive, and those values
which are greater than the truth values are treated as false positive.
\begin{eqnarray*}
\Psi_{TPi}&=&X_i^*(t) \\
\Psi_{FPi}&=&X_i^*(t) - X_i(t) \\
\Psi_{FNi}&=&\Psi_{TNi}=0
\end{eqnarray*}
2. When $X_i(t) > \theta$ and  $X_i^*(t)< \theta  $,
at this point the disaggregation is a false positive.  Then,
\begin{eqnarray*}
\Psi_{FPi}&=&X_i(t) \\
\Psi_{TPi}&=&\Psi_{FNi} =\Psi_{TNi}=0
\end{eqnarray*}
3. When $X_i(t) < \theta$ and  $X_i^*(t) > \theta  $,
at this point the disaggregation is a false negative. Then,
\begin{eqnarray*}
\Psi_{FNi}&=&X_i(t) \\
\Psi_{TPi}&=&\Psi_{FPi} =\Psi_{TNi}=0
\end{eqnarray*}
4. When $X_i(t) < \theta$ and  $X_i^*(t) < \theta  $,
at this point the disaggregation is a true negative.  Then,
\begin{eqnarray*}
\Psi_{TPi}&=&\Psi_{FNi} =\Psi_{FPi} =\Psi_{TNi}=0
\end{eqnarray*}
%In practice, we use different
%$\rho$ and $\theta$ values depending on the
%dataset. For instance, considering the datasets described below,

In our experimental dataset, we set
$\theta=100$ and $\rho=0.2$. 
Although the maximal power consumption of all these devices is 11000W, 
we can still state $\theta < 11000 * 0.1$. 
The reason is that we apply multivariate piecewise motif mining recursively, 
the devices which consume big power are deleted in the first a few rounds. 
Therefore the power noise which is brought by the big power electronic devices 
is greatly decreased. 
