\section{Introduction}
With the advent of modern sensor technologies, 
significant opportunities have emerged to help conserve energy in 
residential and commercial buildings. Moreover, the rapid \emph{urbanization} we are witnessing requires optimized energy distribution. 
Energy disaggregation attempts to 
separate the energy usage 
of each circuit or each electric device in a building 
using only aggregate electricity usage information from 
the whole house meter. 
Usually two-phase or three-phase electric power is 
connected to residential and commercial buildings. 
Similarly, water disaggregation aims to discover each 
water use end by only knowing the 
hot and cold water usage from the whole house water meter.
We generalize these two problems, energy disaggregation and 
water disaggregation, as a multiple-phase data disaggregation problem. 
The aim of this chapter is to identify electrical devices or water use ends from 
two phases of aggregated data. 
Unlike previous work which disaggregate devices
from the sum of multiple phases, 
the time series information from each phase and the correlation of a device between/among phases 
are fully used.  
All of this information enables us to characterize more devices. 
%Our multivariate temporal mining approach extends previous work by combining both electricity and water disaggregation. 
This work makes the following contributions in the field of disaggregation:
\begin{enumerate}
\item It can disaggregate aggregate data from multiple phases.
\item It can separate the continuously variable loads which are mixed in electricity. 
\item This approach can be used for both electricity disaggregation and water disaggregation.
%\item It can be used for supervised learning disaggregation and semi-supervised learning approach, even for un-supervised learning approach. 
\end{enumerate}


