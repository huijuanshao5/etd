\section{Recursive Multivariate Piecewise Motif Mining}
To solve the problem of separating multi-dimensional time series into several time series, 
we propose the approach of recursive multivariate piecewise motif mining. 
Motif mining has been well studied in previous work \cite{motif1} and \cite{motif2}. 
Multivariate or multidimensional motif mining is further extended in \cite{minnen2007improving} and \cite{tanaka2005discovery} and \cite{motifgoal}. 

Motif mining is applied to energy disaggregation in \cite{shao2013temporal}, 
in which discrete on/off events are exploited. 
This research enhances previous work by piecewise motif mining, 
where the on/off event is comprised of several consecutive data points, 
i.e. piecewise, other than individual discrete one.
Also, we use multivariate motif mining to make full use of two or three phases aggregated data. 

The framework of recursive multivariate piecewise motif mining to disaggregation is illustrated in Figure \ref{fig_multivariateMotifming}. 
The input includes the multiple phases aggregated data such as two-phases Mains1 and Mains2, 
and power levels of each device. 
Then we recursively apply piecewise motif mining to two-phases and single phase diffs data.
The first step is to identify electric devices which draw power from both phases.
Generally these devices consume large amount power, such as water heater in blue line.
For devices which draw equal power or disparate power from both phases synchronously. 
%draws equal power from both phases. 
%By comparing the diffs of these two phases, 
%we discover the same power changes during the on/off events. 
%With multivariate motif mining, 
%we can identify these large power consumption devices. 
Secondly, we remove the power consumption of the devices which draw power from both phases.
This procedure decreases the noise interference caused by large power consumption 
and increases the possibility to disaggregate more devices with low-power consumption. 
Then we apply piecewise motif mining to single phase data to separate 
devices which draw power only from it, 
such as humidifier in green. 
\begin{figure}[h]
\centering
\includegraphics[width=0.7\textwidth]{multidisaggfig/RecursiveMultivariateMotifMining.pdf}
\caption{Recursive Multivariate Motif Mining Approach.}
\label{fig_multivariateMotifming}
\end{figure} 

Generally multivariate piecewise motif mining is divided into four steps as shown in Figure \ref{fig_multivariatePiecewiseMotifMining}.
Step 1 is to search for piecewise events from the two-phase or three-phase data.
Next step 2 is to encode events from multiple phases. 
Step 3 aims to mine frequent motifs from the encoded events list.
The last step targets to recover devices from mined motifs. 
\begin{figure}[h]
\centering
\includegraphics[width=6.6in]{multidisaggfig/multivariatePiecewiseMotifMining.pdf}
\caption{Multivariate Piecewise Motif Mining.}
\label{fig_multivariatePiecewiseMotifMining}
\end{figure}

\subsection{Piecewise Motif Mining}
Motif mining aims to uncover the repetitive patterns in time series data. 
It works best for discrete events. Piecewise motif mining is proposed 
for energy disaggregation to detect on/off events. 

\begin{definition}{\textbf{Piecewise Event}}
Given a time series diffs data $y_1, ...., y_{n'}$, where $\forall$ $|y_i| < \eta$. 
A piecewise event is the sum of these $n'$ diffs data, $e= \sum_{i=1}^{n'} y_i$. 
\end{definition}
Each piecewise event corresponds to an on/off event of an electric device. 
The value of $\eta$ is the noise range of each device, 
which is usually less than the 10\% of $|e|$.  

\textbf{Piecewise Events Search from Multiple Phases} \\
Majority of electrical devices which draw power from multiple phases consume larger amount of power, compared to 
electronic devices which connect to single phase. 
To disaggregate such a device, we need to separate on/off events which are yielded by this single device from two-phase or three-phase aggregated data. 
Generally such an electric device draws power from multiple phases synchronously 
and constructs a  pattern. 
Some devices may expend equal power from both phases all the time and the power consumption patterns of both phases are the same.
Others may show different power consumption patterns when drawing power from these two phases. 
%The power consumption from both phases may be exactly synchronized and keep the same all the time. 
%Either, the power consumption drawn from a single device differs much. 
\begin{algorithm}
\caption{Search Synchronized Events from Two-phase Aggregated Diffs Data}
\label{alg_synchronizeEvents}
\begin{algorithmic}[1]
\REQUIRE $2$-phases aggregated diffs data $y_k=y_1^{(k)}, ..., y_n^{(k)}$ and $k=1,2$, %each device's power levels and standard deviation $P_m$, $P_m^{std}$, 
big power consumption threshold $\theta$
%\ENSURE $y = x^n$
\FOR{$i=0: n-1$}
\IF { $|y_i^{(1)}| > \theta$}
\FOR{$j=i-5, i+5$}
\IF{ $|y_j^{(1)}| \in [ |y_j^{(2)}| * 0.8, |y_j^{(2)}| *1.2]$}
\STATE $e_i^{(1)}= e_i^{(1)} + y_j^{(1)}$
\STATE $e_i^{(2)}= e_i^{(2)} + y_j^{(2)}$
\ENDIF
\ENDFOR
\STATE $e_i = e_i^{(1)} +e_i^{(2)}$
\ENDIF
\ENDFOR
\RETURN $e_1, ..., e_i, ..., e_{n'}, \forall e_i > 2*\theta$
\end{algorithmic}
\end{algorithm}
Algorithm~\ref{alg_synchronizeEvents} describes how synchronized events from two phases are revealed.  
This input include the two-phase aggregated diffs data and big power consumption threshold $\theta$. 
This threshold guarantees to discover big power consumption devices firstly. 
We check over the phase-1 diffs data. 
If any absolute value $|y_i^{(1)}|$ is greater than $\theta$, 
both previous and posterior 5 diffs data points from time $i$ are checked. 
For these 10 points values, 
at each time $j$, if the difference between phase-1 $y_i^{(1)}$ and phase-2 $y_i^{(2)}$ is in the range of $0.2*|y_i^{(1)}|$,  
we assert that the diffs data points from these two phases are relatively the same and synchronized. 
The synchronization implies that 
these two same power consumption comes from a single device. 
Therefore we sum the synchronized power level diffs data and compute the power consumption at time $i$ as $e_i$.  
$e_i>0$ denotes an on event and $e_i<0$ means an off event of certain device.

Now we transfer these two-phase diffs data into 
an ordered on/off events list $e_1, ..., e_{n'}$.

Next we apply motif mining to this events list. 
By matching the devices which consume power bigger than $2*\theta$, 
we can separate all devices which draw equal amount of power from two phases.  
%Since we already know the power levels of each device, 
%we just choose those devices which include power levels bigger than $2*\theta$. 
%By applying motif mining, 
%we can separate all devices which consume two-phase power greater than $\theta$ equally. 
%For dataset Study10, we set $\theta=500W$. 
%This approach helps us to discover two devices waterHeater2 and heatingIndoor. 
%All of the on/off events of these two devices are found out. 

\subsection{Encoding Events From Multiple Phases}
After deleting all the synchronized events from the phase-1 and phase-2, 
we apply multivariate piecewise motif mining to the remaining phase-1 and phase-2 diffs data. 
These devices consume large power 
yet they draw power from two phases synchronously but unequally. 
They show different power drawing patterns from these two phases.  
We encode these two-phase diffs data which occur at the same as a new event $e$. 
Figure \ref{fig_eventEncoding} gives an example on how the events from two-phase 
are encoded. 
We extract an event which consumes power greater than $\theta$, 
then we check 5 more data points before and after it. 
The total 11 data points relevant to this event in Main1.diff are [0, 0, -18, 18, 1093, 1830, -196, -68, -37, -36, 0]. 
The concurrent events listed in Main2.diff are [0, 0, 0, 18, 9, 1946, 440, -51,-36,-36,0]. 
Since the events at the peak occur in two-phase as $(1830, 1946)$, 
and the difference of these two powers are in the $0.2*1830$ range, 
we consider that these two changes may come from a single device. 
When looking insight into these two vectors, 
we observe that the sum of the changes of phase 1 is 2604W, and the sum of the changes of phase 2 is 2290W. 
They are in the same range, i.e. $2604*0.8 < 2290$. 
Therefore, we declare that the power changes from these two phases should definitely come from a single device. 
We select two of them and encode them as $e_{1'}=(1093, 9)$, $e_{2'}= (1830, 1946)$
\begin{figure}[h]
\centering
\includegraphics[width=0.6\textwidth]{multidisaggfig/synchronizeDifferentEventEncoding.pdf}
\caption{Encoding Events from Multiple Phases.}
\label{fig_eventEncoding}
\end{figure}

The piecewise events for this single device is $e= [e_{1'}, e_{2'}]$. 
Applying frequent motif mining, 
we separate this large power consumption device which draw power from two phases unequally. 



