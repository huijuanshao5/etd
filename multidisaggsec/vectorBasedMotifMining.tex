\subsection{Vector-based Motif Mining}
Motif mining has been studied in previous research \cite{motif1} and \cite{motif2}. 
After extracting the usage levels, standard deviation of usage levels,
and on startup duration, 
we apply a vector-based motif mining to 
disaggregate each device from aggregated data. 
This vector-based motif mining is comprised of two steps:  
segmenting the piece-wise usage vector features from 
the aggregated data, 
and applying motif mining on those above vectors. 

\subsubsection{Piece-wise Vector Selection}
Algorithm \ref{alg_vectorSelection} describes how the process of 
the usage levels of each device are selected from aggregated data. 
The input is the output of Algorithm \ref{alg_extractFeatures}, 
i.e. for device $m$, usage levels $P_m$,  
the usage level's standard deviation $P_m^{std}$, 
and on/off duration $D_m$. 
During startup, the usage level of the same device may jump several times before 
reaching a steady state. 
We set a minimal usage level $P_{min}$ which satisfies the usage levels for all devices.  
This minimal usage value $P_{min}$ is caused by the noise of device, 
therefore it should be large enough to erase the influence of noise. 
For specific dataset study10, $P_{min}$ sets to be $80W$ in aggregated power data. 
Similarly, according to the sampling interval of the aggregated data, 
we set a max time gap $D_{max}$ between any two points in the aggregated data. 

To judge whether a vector, i.e. continuous usage level changes, 
come from a single device, 
we evaluate the values of consecutive usage elves changes and the duration gap between them. 
The procedure is shown between line 3-10.
For instance, if we have a piece-wise aggregated data as $y_1, y_2, y_3, y_4$, 
the absolute diffs values between any two consecutive points 
$y_2-y_1$, $y_3-y_2$, $y_4-y_3$ are greater than $P_{min}$, 
and all the corresponding gaps between any two consecutive points $t_2-t_1$, $t_3-t_2$, $t4-t3$ 
are less than $D_{max}$, 
we assume that these piece-wise usage levels $y_1,..., y_4$ come 
from a single device. 
If any of the usage level changes such as $|y_2 - y_3| < P_{min}$ or 
then, the usage level changes are separated into 
two piece-wises $y_1, y_2$ and $y_2, y_3, y_4$. 
On the other hand, 
if the gap between any two consecutive points such as $y_3.{t}-y_2.{t} > D_{max}$, 
we assume that data points between $y_2$ and $y_3$ are missing.
Each discovered piece-wise vector is classified into the most similar group $l_m$ as 
for device $m$. 
Then we return all the piece-wise vectors list $(l_1,...,l_k)$.
%The drawback of this assumption is that we cannot distinguish devices 
%which power on and off for the same time. 

This piece-wise vector calculation is different from the diffs calculation from other papers 
such as \cite{shao2013temporal} and \cite{kolter2012aistat}. 
Its merit is very obvious that many devices start up not at a fixed power level changes. 
For instance, the heatingIndoor device may jumps to a steady state 
$11000W$ once in two seconds, or jump twice to $8000W$ in 2 seconds then to 
$11000W$ in five seconds. 
These piece-wise vectors are especially useful for identifying continuously variable loads or 
water end use.  

\begin{algorithm}
\caption{Piece-wise vector computation procedure}
\begin{algorithmic}[1] 
\label{alg_vectorSelection}
\REQUIRE  aggregated power $y = (t_1, y_1),..., (t_N, y_N)$, usage level range $P_{min}$, gap range $D_{max}$
%\ENSURE $y = x^n$
\STATE $i \leftarrow 0$, $l_k \leftarrow \phi$
\STATE $n \leftarrow 0$
\WHILE{$n \neq N$}
\IF{$t_{n+1}- t_{n} < D_{max}$ and $|y_n-y_{n+1}| > P_{min}$}
\STATE classify $\{t_n, y_n, ..., t_{n+1}, y_{n+1}\}$ into the most similar category  $l_k$
\ELSE
\STATE $k \leftarrow k + 1$
\ENDIF
\STATE $n \leftarrow n+1$
\ENDWHILE
\RETURN $l_1,..., l_k$, where $l_k= [t_k, y_k, ..., t_{k+k'}, y_{k+k'}]$
\end{algorithmic}
\end{algorithm}

After we get the piece-wise diffs data, which is composed of lists of consecutive time and usage levels 
$[t_1, y_1, t_2, y_2], ..., [t_{k+k'}, ..., y_{k+k'}]$ where $t_k$ is the time stamp for 
the corresponding aggregated value $y_k$, and $k$ is the number of diffs clusters, 
$k'$ is the number of consecutive values in cluster $k$. 

In order to disaggregate each device from the aggregated diffs usage levels clusters, 
we apply vector-based classification approach as shown in Algorithm \ref{alg_disaggDevices}.
We classify the vectors $l_k$ into device $m$ where the usage levels, and duration are most similar. 
%It extends previous motif mining approach by adopting piece-wise vectors other than discrete events. 

\begin{algorithm}
\caption{Vector-based Classification Procedure}
\begin{algorithmic}[1]
\label{alg_disaggDevices}
\REQUIRE piece-wise vectors $l_1,...l_k$ where $l_k= [t_k, y_k, ..., t_{k+k'}, y_{k+k'}]$, 
usage level of device $m$, each device's power levels, standard deviation and startup duration $P_m$, $P_m^{std}$, $D_m$,
%\ENSURE $y = x^n$
\FOR{$m \in (1, M)$}
\WHILE{$k \neq K$}
\IF { \forall $y' \in{ (y_k,...y_{k+k'}) }$, $P_m - P_m^{std} < y' < P_m + P_m^{std}$ and $D_m < t_{k+k'}- t_k$}
\STATE $S_m \leftarrow l_k $
\ENDIF
\STATE $ k \leftarrow k+1$
\ENDWHILE
\ENDFOR
\RETURN diffs lists set for each device $S_m$
\end{algorithmic}
\end{algorithm}


This matching process can disaggregate 
continuously variable loads such as HVAC, 
which are very difficult to disaggregate because 
their power consumption is composed of several power level changes 
other than individual discrete value. 
Whenever it starts up, it will takes several seconds or minutes to stay in a 
steady state. 
As long as we input the startup duration $D_m$, 
we can find them from the aggregated data. 

Also, we integrate this classification algorithm with motif mining approach to these piece-wise vector lists.
Motif mining gets the diffs vectors and 
process the vector event, which is composed a series of point events and duration instead of the a single discrete point event. 
%The match process is not discrete value, but diffs lists. 
%In this paper, we propose a temporal mining approach to disaggregate 
%such continuously variable loads which are mixed with 
%discrete power levels. 
%The results show that, this searching approach can help disaggregate 
%HVAC. 

\subsubsection{Dynamic Time Warping Subsequence Search or DTW-based Motif Mining}
Dynamic time warping subsequence search has been studied well in previous work \cite{rakthanmanon2012searching}.
We apply dynamic time warping subsequence search to find some continuously variable load which 
starts up for more than 1 minutes, or water use end which uses water in different water flow rate. 
Usually this kind of electric device shuts down in a very short period of time, 
but it takes long time to start. 
The pure power levels, startup duration features are not sufficient to 
discover it especially when it is mixed with other devices. 
To separate its feature from aggregated data, 
we need to calculate the dynamic time warping distance between the startup shape $P_1, ...., P_{2\Delta t}$ and the 
shape features of each device. 
Dynamic time warping subsequence search helps us find the top-k similar startup shapes.

%Figure \ref{fig_dtw} gives an example on how the 
%device heatingIndoorUnit and heatingOutdoorUnit 
%is discovered. 
%\begin{figure}[!t]
%\centering
%\includegraphics[width=2.5in]{figs/dtw.png}
%\caption{Discover HeatingIndoorUnit and HeatingOutdoorUnit in Time Series. }
%\label{fig_dtw}
%\end{figure}

%\subsection{Multivariate Motif Mining}
%Water disaggregation is similar to electricity disaggregation. 
%The difference is that the water usage comes from 
%two main water pipe, cold water and hot water. 
%Our problem becomes disaggregate several devices from 
%two time series cold water flow rate and hot water flow rate. 

%We use multivariate time series motif mining \cite{Deb KDD} approach to 
%water disaggregation. 
%An example of multivariate motif mining is shown in Figure \ref{fig_multivariateMotif}.
%\begin{figure}[!t]
%\centering
%\includegraphics[width=2.5in]{figs/multiMotif.png}
%\caption{Multivariate Motif Mining Approach.}
%\label{fig_multivariateMotif}
%\end{figure}

%The process of multivariate motif mining is described in Algorithm \ref{alg_multivariateDisagg}.
%\begin{algorithm}
%\caption{Disaggregate each device from diffs lists $l_1,...l_k$ where $l_k= [t_k, y_k, ..., t_{k+k'}, y_{k+k'}]$}
%\begin{algorithmic} 
%\label{alg_multivariateDisagg}
%\REQUIRE aggregated power diffs lists $l_1,...l_k$ where $l_k= [t_k, y_k, ..., t_{k+k'}, y_{k+k'}]$, 
%power level of device $m$, each device's power levels, standard deviation and startup duration $P_m$, $P_m^{std}$, $D_m$,
%\ENSURE $y = x^n$
%\FOR{$m \in (1, M)$}
%\WHILE{$k \neq K$}
%\IF { \forall $y' \in{ (y_k,...y_{k+k'}) }$, $P_m - P_m^{std} < y' < P_m + P_m^{std}$ and $D_m < t_{k+k'}- t_k$}
%\STATE $S_m \leftarrow l_k $
%\ENDIF
%\STATE $ k \leftarrow k+1$
%\ENDWHILE
%\ENDFOR
%\RETURN diffs lists set for each device $S_m$
%\end{algorithmic}
%\end{algorithm}