\section{Prior Work}
%%% electricity disaggregation prior work
Electricity disaggregation uses the electricity consumption level at the main entry into a building or house 
to infer whether a device inside the building is on or off. 
The features used include initial real power and reactive power~\cite{hart1992} from a dataset which is recorded in 
a low-frequency range. 
With advances in electrical meter technology and the availability of less expensive meters, 
more and more features are being extracted from the high-frequency data set and used for disaggregation, such as 
the transient state generated when a device turns on or off~\cite{shaw2000PhdThesis},
the raw current waveform~\cite{srinivasan2006neural}, the voltage waveform~\cite{lam2007novel}, 
the transform of the current waveform~\cite{chan2000harmonics}, 
and harmonics of non-linear devices~\cite{chan2000harmonics}. 
Even on-AC power features such as power line noises~\cite{patel2007flick}
are exploited jointly with AC power features like  
time of day, and device correlations~\cite{kim2011unsupervised}
in modern systems.

%Algorithms applicable to energy disaggregation
%can be categorized into supervised learning algorithms, 
Increasingly,  
research is being focused on unsupervised learning and semi-supervised learning algorithms because  
these algorithms do not require the power consumption of 
each device,   
and the power or water usages of individual devices are very difficult to obtain. 
%The applied data mining algorithms are composed of
%supervised learning algorithms, unsupervised learning algorithms
%and semi-supervised learning algorithms.
%The supervised learning algorithms include
%kNN methods~\cite{shaw2000PhdThesis},
%support vector machines~\cite{patel2007flick},
%neural networks~\cite{roos1994using},
%genetic programming~\cite{baranski2004genetic},
%sparse coding~\cite{kolter2010sparse}, 
%as well as
%combinations of supervised learning algorithms~\cite{nakano2007non}. 
%Optimization algorithms used in the area of energy disaggregation have
%been drawn from integer programming~\cite{suzuki2008nonintrusive}, 
%dynamic programming~\cite{baranski2004detecting}, and
%the Viterbi algorithm~\cite {zeifman2011viterbi}.
It is only in
 the last few years that 
unsupervised learning algorithms
have been used, including
hierarchical clustering~\cite{gonccalves2011unsupervised},
factorial hidden Markov models (FHMMs)~\cite{kim2011unsupervised},
additive factorial approximate MAPs (AFAMAP)~\cite{kolter2012aistat}, 
difference FHMMs~\cite{parson2012nonintrusive}, 
and motif mining~\cite{shao2013temporal}.
Semi-supervised learning 
algorithms~\cite{lam2007novel,johnson2012bayesian} have also
been proposed.
In this chapter, we assume the number of devices 
and the number of power level states of each device 
are known. Hence, we formalize the disaggregation 
as a semi-supervised problem and 
provide solutions to the following three challenging problems.
%The field of energy disaggregation has evolved
%over the last twenty years. While
%some applications have achieved
%qualified success, there are several challenging
%problems to be addressed before energy disaggregation can be
%used more widely. Some of these problems include:
\begin{enumerate}
%\item The number of devices is typically unknown and can only be
%approximately estimated based on background information.
%\item The number of power level states of each device is unknown.
%Some devices such as lights may have only two steady states, viz. on and off.
%Other devices  have several steady states.
%For example, a microwave can operate in the states of defrost,
%heat with low power, or heat with high power. Estimating
%the exact number of states of a device is a hard problem.
\item Several devices may have the same real power, and
it is difficult to distinguish these devices using only the recorded aggregated
power time stamp.
%For example, a light and a monitor could consume the same amount of real
%power (e.g., around 38W). With more devices that share the same real power,
%additional features are necessary to disambiguate among them.
\item Many devices may turn on or off at the same time.
%A PC and printer likely turn on and off together,
%thus making it difficult to separate them from the aggregated power profile.
\item Instead of having a discrete range of power
levels, there are devices whose power consumption levels
   vary gradually, e.g.,
  variable speed devices (VSD) and lights with dimmers.
Once their power usage is aggregated with that from other devices,
disaggregation becomes increasingly difficult.
%\item Some devices are always on and seldom operated by
%users. Because the operations on these devices are
%rare, it is hard to identify these devices from prior historical data.
\end{enumerate}
Since obtaining a low-frequency dataset is more practical in real buildings, 
we focus mainly on real power, which can be easily extracted 
from a low-frequency dataset.  
%In this paper, we use the low frequency dataset therefore
%only the real power feature is used. 
%We assume that the number of devices and their real power 
%levels are known in advance. 
%We formalize this energy disaggregation as a semi-supervised problem
%and provide solutions to the above three sub-problems 2-5. 

%Though electricity disaggregation has been researched over decade, 
%researches on water disaggregation just emerges in recent years. 
%Supervised, semi-supervised and unsupervised learning approaches have been intensively
%studied in electricity disaggregation. In contrast, 
Water use disaggregation has emerged in recent years, and so far 
the applied algorithms are 
limited to supervised learning algorithms~\cite{carboni2016contextualising}. 
This chapter proposes water disaggregation as a semi-supervised learning algorithm 
by presuming that we know the number of water use ends and the water usage level of each 
water user end.  

%%%%water disaggregation prior work
\begin{comment}
Similar to electricity disaggregation, water disaggregation uses the total cold water and hot water 
usage at home to deduce when an water end uses. 
Sensors are installed on the whole house cold water and hot water pipes to record the water pressure or water flow rate \cite{carboni2016contextualising}. 
Prior work on water disaggregation focuses on supervised disaggregation. 
 HydroSense \cite{froehlich2009hydrosense} firstly proposed to exploit water pressure for non-intrusive water 
 disaggregate. Later \cite{larson2012disaggregated} extends its approach as hidden Markov model. 

%The water flow rate data are collected by installing sensors contacting water directly and logs the 
%data in the unit of liter per minute. The disaggregation approaches are similar to the approaches for 
%water pressure data. 
Features such as a water usage cycle of a device, total amount of water, 
the usage duration of a device,  the interval between two usages of the same device, 
time of the day, weekend or not are extracted in the research of \cite{fontdecaba2013approach}. 
Then a probabilistic model which computes the likelihood of water end use categories is applied 
for classification. 

A combination approach of hidden Markov model and dynamic time warp is 
proposed in \cite{nguyen2013development} for water end use classification. 
Majority of the water end uses are identify by hidden Markov model 
after supplying their water usage changes. 
Additional, dynamic time warping is applied to categorize certain devices such as dish washer 
 and clothes washer. 
 A deep sparse coding-based recursive disaggregation model is proposed in \cite{dong2013deep}. 
 It uses the water consumption feature as the base for supervised disaggregation. 
 This approach is highly related to the deep sparse coding approach to electricity disaggregation \cite{kolter2010sparse}.
 A convex optimization approach is introduced for both electricity and water disaggregation in \cite{pigaconvex}. 
 It utilizes piece-wise water consumption as a feature as input, then applies the convex optimization approach 
 to minimize the euclidean distance between the aggregated data and sum of disaggregated data. 
 In the work of \cite{seppelt2012analysis}, 
 the on operations of each water end use are split into several segments as features.
 Then a hidden Markov model uses each feature as an event for deciding which device an event from 
 aggregated data it should classify into. 
 
The challenges we are facing for supervised water disaggregation are similar to electricity disaggregation as following. 
\begin{enumerate}
\item Variable water usage pattern for the same water end use: for example, 
a sink in a kitchen uses both cold water and hot water. Its water usage varies 
in a great range, a large amount of water to wash foods or a very small amount to wash hands. 
This variety of water usage makes it difficult to find a fixed pattern in kitchen. 
Note that this variable water usage device is similar to some electric devices like continuous variable load in electricity disaggregation. 
\item The overlapped usage of different water end uses: 
For instance, people may use toilet sink before or after toilet. 
The overlap of the these two devices causes difficulty in separating them 
especially when the sink water usage is very irregular. 
\item Similar water usage patterns from different water end uses: for instance, two sinks are located in different rooms. 
Even if they are often used in different periods of time, 
it is very hard to distinguish them. 
\end{enumerate}
For unsupervised water disaggregation, more challenges exist, 
such as the number of water end uses, and the water usage of each water end use. 

This is the first work to disaggregate electricity and water under the same framework. 

\end{comment}
 
