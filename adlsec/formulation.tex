\section{Problem Formulation}
Before formalizing the problem statement, 
we introduce several concepts and notations first. 

\textit{Episode} An episode is a pattern which can be explained meaningfully. For instance, we represent 'S' as sleep, 
'K' as kitchen, and 'Z' as going out. 
If an episode $S \rightarrow K \rightarrow Z$ is found, 
the story is described as a person getting up, 
going to the kitchen for breakfast, and then leaving the house. 
An episode $\alpha$ is composed 
of a series of ordered events
$\alpha=\langle E_1,..,E_t,...E_n \rangle$, 
where $E_t$ denotes that $E$ occurs at time $t$.  
These events may be the point event or dwelling event. 
The dwelling event
has a start time $E.start$ and end time $E.end$. 
In this chapter  $E$ denotes a dwelling event and 
represents which room a person stays 
inside the building i.e is \emph{occupied}. 
For example we if $Z$ is used to show a room to be unoccupied. 
$Z.start$ means when the person goes out of home and 
$Z.end$ means when the person comes back. 

\textit{EGH} Episode generative HMM model is a 
type of HMM model which connects each episode $\alpha$ 
with a special HMM model $\Lambda_\alpha$. 
The uniqueness of the EGH is that 
the transition matrix and 
emission matrix is only decided by a noise parameter $\eta$. 
The value of $\eta$ is computed based on the frequency of the 
corresponding episode $\alpha$. 

We formulate the occupancy prediction problem as one of 
episode mining and event type time prediction problem. 

\textit{Problem Statement} Given a sequence of the room occupation 
events stream \textit{s} with finite events symbols $\varepsilon$, 
and the target un-occupancy event \textit{Z}, can we
find the frequent patterns $\{\alpha_1,...\alpha_n\}$ 
and can the corresponding episode generative HMM 
$\{\Lambda_{\alpha_1}, ..., \Lambda_{\alpha_n}\}$, 
predict when 
the person leaves $Z.{start}$, 
and when the person comes back $Z.{end}$.?

Next, 
we first discuss the time-gap constraint episode mining 
model and the mixture model, 
and how to predict the target event in section 4. 
Then, 
in section 5 we will show 
the experimental results. 

\iffalse
The training phase is to mine the frequent episodes and connect it with HMM model
as described in \cite{laxman2008stream}. We modify the constraints during episode mining
and keeps the connections between the frequent episodes and HMM the same. 
The training algorithm describes in Algorithm \ref{alg1}.
\fi

%% algorithm 01
\renewcommand{\algorithmicrequire}{\textbf{Input:}}
\renewcommand{\algorithmicensure}{\textbf{Output:}}

\begin{algorithm}
\caption{Trainng Algorithm}
\label{alg1}
\begin{algorithmic} [1]
\REQUIRE training events stream, $S_t=<\hat{E_1}, ..., \hat{E_n}>$; 
target event type Z; size of symbols $\varepsilon$ M; 
window size  \textit{W} preceding the target event Z
\ENSURE Generative model, $\Lambda_Z$ 

\COMMENT{re-organize data}\\
\STATE split data into each day
\STATE rename the symbol based on 15-minutes chuck
\STATE Initialize the partial stream $D_Z= \phi$ before $Z$

\FOR{the data of each day}
\IF{$\exists$ t that $\hat{E_t}=Z$}
\STATE add $S_t=<\hat{E_{t-W}}, ..., \hat{E_{t-1}}>$ to $D_Z$
\ENDIF
\ENDFOR

\COMMENT{build mixture episode generative HMM model}
\FOR{$1<w<W$}
\STATE mine the frequent episodes $\{\alpha_1, ..., \alpha_J\}$ 
\ENDFOR

\STATE connect each episode $\alpha_j $ with a HMM model EGH $\Lambda_{\alpha_j}$ 

\STATE generate episode generative HMM (EGH) mixture model $F(\Lambda)= \sum_1^J \theta_j\Lambda_{\alpha_j}$, where $1 \leq j \leq J$  
with EM algorithm

\STATE Output $\Lambda_Z=\{  (\Lambda_{\alpha_j}, \theta_j ) , j=1,...,J \}$

\end{algorithmic}
\end{algorithm}
\iffalse
From line 1-2, the symbols are reorganized according to the 15-minutes chuck of each day. 
If there are more than one symbols in a 15-minutes chuck, 
the symbol with longest duration event will be chosen. 
This step is to emphasize there is only one place a person stays in the room. 
From line 4-7, the W length sequence before the target symbol Z firstly appears is chosen 
for each day. 
From line 9-11, the frequent episodes are mined. 
From line 12-14, for each episode, a HMM model named EGH is generated and connected. Then a mixture model 
with all these EGHs are calculated with EM algorithm. 

The event time prediction algorithm Algorithm \ref{alg2} is composed of two stages. 
In the first stage from line 1-15,
we could predict whether the next event $E_{t+q}$ is the target event Z or not. 
For each $t$ is greater than W, intercept the sequence of length $W$ 
$X=<E_{t-W}, ..., E_t>$ in line 2. 
If Z is inside the X, then calculate the possible largest Z index $t_{max}$ inside X. 
In lines 4-6 the mixture model is calculated whether the next time $t+1$ would be the target event or not.
In line 8, all the frequent episodes will be checked whether it is inside the sequence X after $t_{max}$
If there is any frequent episode inside X, 
then we could predict the next event $E_{t+1}=Z$. 
That is, if any of the frequent episode happens 
and the value is greater than the threshold, 
then the next event is Z. 

In the second stage, the exact leave and back time prediction of Z is described from line 16-25. 
If a person gets up before sleep, 
generally we don't know when he will leave or come back. 
In this case, the predicted time is based on the average value of the past leave and back time. 
If a person already gets up, 
we mine the episode patterns, 
and predict according to the mixture model. 
This will be detailedly explained in \ref{alg22}.
If a person already leaves, 
the back time prediction is also based on the historical past time with 
a constraint on the leaving time. 
For a person may come back home and stay home for sometimes then go out again, 
we could repeat the second stage until the end of day. 
\fi
%% algorithm 02
\renewcommand{\algorithmicrequire}{\textbf{Input:}}
\renewcommand{\algorithmicensure}{\textbf{Output:}}

\begin{algorithm}
\caption{Prediction Algorithm}
\label{alg2}
\begin{algorithmic} [1]
\REQUIRE Partial day's event stream, $s=<E_1, ..., E_t, ...>$; 
target event type Z; size of symbols $\varepsilon$ M; 
window size  \textit{W} preceding the target event Z;
generative model $\Lambda_Z=\{  (\Lambda_{\alpha_j}, \theta_j ) , j=1,...,J \}$;
threshold $\gamma$

\ENSURE Predict $Z_{leave}$ and $Z_{back}$ if $E_{t+1}=Z$ or $E_{t+1}\neq Z$ 

\FOR{$W \leq t \leq len(s)$}
% get the stream inside a window W
\STATE set $X=<E_{t-W}, ..., E_t>$
\STATE $t_Z=0$

\COMMENT{the probability of X given the mixture model } \\
\IF{$P[X|\Lambda_Z] =\sum_1^J \theta_j\Lambda_{\alpha_j} \geq \gamma$} 

\IF{ $Z \in X$ }
\STATE Set $t_Z$ to largest  t, where $E_{t_{max}}=Z$ and $ t-W \leq t_{max} \leq t$
\ENDIF

\IF{$\exists \alpha \in \{\alpha_1,..., \alpha_J\}$ in X after $t_{max}$}
\STATE Predict $E_{t+1}=Z$
\STATE Record time of each symbol in these frequent episodes \\
%\COMMENT{calculate $\Delta t$ in Algorithm \ref{alg22}}
\ELSE
\STATE Predict $E_{t+1} !=Z$
\ENDIF
\ENDIF %end of the gamma
\ENDFOR

\IF{$E_{t+1} =Z$}

\IF{$t_Z$ happens before a person gets up}
\STATE $S_{leave}= \frac{\sum_1^J Z_{leave}}{J}$ where $Z_{leave}= t_{Z_{first}} \in \alpha_j$
\STATE $S_{back}= \frac{\sum_1^J Z_{back}}{J}$ where $Z_{back}= t_{Z_{last}} \in \alpha_j$
\ELSIF{$t_Z$ happens before a person gets up}
\STATE $S_{leave}$ and $_{back}$ are predicted by mixture models
\ELSE
\STATE $S_{back}= \frac{\sum_1^K Z_{back}}{K}$ where $Z_{leave}= t_{Z_{last}} \in \alpha_j$ 
and $Z_{leave}$ has a noise $\epsilon$

\ENDIF

\ENDIF
\STATE Output $E_{t+1}$, $S_{leave}$, $S_{back}$

\end{algorithmic}
\end{algorithm}




