\section{Problem Formulation}
Given $M$ time series, 
each time series $X^{(m)}=X^{(m)}_1, ..., X^{(m)}_t, ..., X_T^{(m)}$ 
represents a sequence of room occupancy of person $m$ inside 
a home over $K$ days, 
where $X_t\in s$ denotes that $X$ belongs to a finite room set $s$ at the sequence number of $t$, 
and $m\in\{1,...,M\}$. 
%Each $X$ is has a start time $X.start$ and an end time $X.end$.  
Let $Z$ denote that the home is unoccupied and $Z\in s$.  %$Z=0$ if a house is unoccupied; $Z=1$ if the house is occupied.  
%Its meaning equals to certain symbols from set $s$. 
We predict whether person $m$ stays at home the rest of a day from time $T$,  i.e. during $T+1$, $T+2$, ..., $\Delta T$, 
 
\begin{equation}
\hat{Y}^{(m)}=\hat{Y}_{T+1}^{(m)},...,\hat{Y}^{(m)}_{T+\Delta T}
\end{equation}
where $Y_{T+\Delta t }^{(m)}=Z$ if person $m$ does not stay at home at time $T+\Delta t $;
otherwise, $Y_{T+\Delta t}^{(m)} \neq Z$. 
%The un-occupancy of the whole house is the intersection of un-occupancy of all people. 
If any person $m$ stays at home $Y_{T+\Delta t}^{(m)} \neq Z$, then this house is occupied $Y_{T+ \Delta t} \neq Z$. 


\iffalse
Given $M$ time series, 
each time series $X^{(m)}=X^{(m)}_1, ..., X^{(m)}_t, ..., X_T^{(m)}$ 
represents a sequence of room occupancy of person $m$ inside 
a home over $K$ days, 
where $X_t\in s$ denotes that $X$ belongs to a finite room set $s$ at the sequence number of $t$. 
Each $X$ is has a start time $X.start$ and an end time $X.end$.  
Let $Z$ denote that the home is unoccupied. 
It equals to certain symbols from set $s$. 
Also, given a time gap from current time $\Delta t$, 
predict whether $Z$ appears at time  $T+\Delta t$

 formulate the occupancy prediction problem as one of 
episode mining and event type time prediction problem. 
\textit{Problem Statement} Given a sequence of the room occupation 
events stream \textit{s} with finite events symbols $\varepsilon$, 
and the target un-occupancy event \textit{Z}, can we
find the frequent patterns $\{\alpha_1,...\alpha_n\}$ 
and can the corresponding episode generative HMM 
$\{\Lambda_{\alpha_1}, ..., \Lambda_{\alpha_n}\}$, 
predict when 
the person leaves $Z.{start}$, 
and when the person comes back $Z.{end}$.?

The training phase is to mine the frequent episodes and connect it with HMM model
as described in \cite{laxman2008stream}. We modify the constraints during episode mining
and keeps the connections between the frequent episodes and HMM the same. 
The training algorithm describes in Algorithm \ref{alg1}.
\fi

%% algorithm 01
\renewcommand{\algorithmicrequire}{\textbf{Input:}}
\renewcommand{\algorithmicensure}{\textbf{Output:}}

\begin{algorithm}
\caption{Trainng Algorithm}
\label{alg1}
\begin{algorithmic} [1]
\REQUIRE training events stream, $S_t=<\hat{E_1}, ..., \hat{E_n}>$; 
target event type Z; size of symbols $\varepsilon$ M; 
window size  \textit{W} preceding the target event Z
\ENSURE Generative model, $\Lambda_Z$ 

\COMMENT{re-organize data}\\
\STATE split data into each day
\STATE rename the symbol based on 15-minutes chuck
\STATE Initialize the partial stream $D_Z= \phi$ before $Z$

\FOR{the data of each day}
\IF{$\exists$ t that $\hat{E_t}=Z$}
\STATE add $S_t=<\hat{E_{t-W}}, ..., \hat{E_{t-1}}>$ to $D_Z$
\ENDIF
\ENDFOR

\COMMENT{build mixture episode generative HMM model}
\FOR{$1<w<W$}
\STATE mine the frequent episodes $\{\alpha_1, ..., \alpha_J\}$ 
\ENDFOR

\STATE connect each episode $\alpha_j $ with a HMM model EGH $\Lambda_{\alpha_j}$ 

\STATE generate episode generative HMM (EGH) mixture model $F(\Lambda)= \sum_1^J \theta_j\Lambda_{\alpha_j}$, where $1 \leq j \leq J$  
with EM algorithm

\STATE Output $\Lambda_Z=\{  (\Lambda_{\alpha_j}, \theta_j ) , j=1,...,J \}$

\end{algorithmic}
\end{algorithm}
\iffalse
From line 1-2, the symbols are reorganized according to the 15-minutes chuck of each day. 
If there are more than one symbols in a 15-minutes chuck, 
the symbol with longest duration event will be chosen. 
This step is to emphasize there is only one place a person stays in the room. 
From line 4-7, the W length sequence before the target symbol Z firstly appears is chosen 
for each day. 
From line 9-11, the frequent episodes are mined. 
From line 12-14, for each episode, a HMM model named EGH is generated and connected. Then a mixture model 
with all these EGHs are calculated with EM algorithm. 

The event time prediction algorithm Algorithm \ref{alg2} is composed of two stages. 
In the first stage from line 1-15,
we could predict whether the next event $E_{t+q}$ is the target event Z or not. 
For each $t$ is greater than W, intercept the sequence of length $W$ 
$X=<E_{t-W}, ..., E_t>$ in line 2. 
If Z is inside the X, then calculate the possible largest Z index $t_{max}$ inside X. 
In lines 4-6 the mixture model is calculated whether the next time $t+1$ would be the target event or not.
In line 8, all the frequent episodes will be checked whether it is inside the sequence X after $t_{max}$
If there is any frequent episode inside X, 
then we could predict the next event $E_{t+1}=Z$. 
That is, if any of the frequent episode happens 
and the value is greater than the threshold, 
then the next event is Z. 

In the second stage, the exact leave and back time prediction of Z is described from line 16-25. 
If a person gets up before sleep, 
generally we don't know when he will leave or come back. 
In this case, the predicted time is based on the average value of the past leave and back time. 
If a person already gets up, 
we mine the episode patterns, 
and predict according to the mixture model. 
This will be detailedly explained in \ref{alg22}.
If a person already leaves, 
the back time prediction is also based on the historical past time with 
a constraint on the leaving time. 
For a person may come back home and stay home for sometimes then go out again, 
we could repeat the second stage until the end of day. 
\fi
%% algorithm 02
\renewcommand{\algorithmicrequire}{\textbf{Input:}}
\renewcommand{\algorithmicensure}{\textbf{Output:}}

\begin{algorithm}
\caption{Prediction Algorithm}
\label{alg2}
\begin{algorithmic} [1]
\REQUIRE Partial day's event stream, $s=<E_1, ..., E_t, ...>$; 
target event type Z; size of symbols $\varepsilon$ M; 
window size  \textit{W} preceding the target event Z;
generative model $\Lambda_Z=\{  (\Lambda_{\alpha_j}, \theta_j ) , j=1,...,J \}$;
threshold $\gamma$

\ENSURE Predict $Z_{leave}$ and $Z_{back}$ if $E_{t+1}=Z$ or $E_{t+1}\neq Z$ 

\FOR{$W \leq t \leq len(s)$}
% get the stream inside a window W
\STATE set $X=<E_{t-W}, ..., E_t>$
\STATE $t_Z=0$

\COMMENT{the probability of X given the mixture model } \\
\IF{$P[X|\Lambda_Z] =\sum_1^J \theta_j\Lambda_{\alpha_j} \geq \gamma$} 

\IF{ $Z \in X$ }
\STATE Set $t_Z$ to largest  t, where $E_{t_{max}}=Z$ and $ t-W \leq t_{max} \leq t$
\ENDIF

\IF{$\exists \alpha \in \{\alpha_1,..., \alpha_J\}$ in X after $t_{max}$}
\STATE Predict $E_{t+1}=Z$
\STATE Record time of each symbol in these frequent episodes \\
%\COMMENT{calculate $\Delta t$ in Algorithm \ref{alg22}}
\ELSE
\STATE Predict $E_{t+1} !=Z$
\ENDIF
\ENDIF %end of the gamma
\ENDFOR

\IF{$E_{t+1} =Z$}

\IF{$t_Z$ happens before a person gets up}
\STATE $S_{leave}= \frac{\sum_1^J Z_{leave}}{J}$ where $Z_{leave}= t_{Z_{first}} \in \alpha_j$
\STATE $S_{back}= \frac{\sum_1^J Z_{back}}{J}$ where $Z_{back}= t_{Z_{last}} \in \alpha_j$
\ELSIF{$t_Z$ happens before a person gets up}
\STATE $S_{leave}$ and $_{back}$ are predicted by mixture models
\ELSE
\STATE $S_{back}= \frac{\sum_1^K Z_{back}}{K}$ where $Z_{leave}= t_{Z_{last}} \in \alpha_j$ 
and $Z_{leave}$ has a noise $\epsilon$

\ENDIF

\ENDIF
\STATE Output $E_{t+1}$, $S_{leave}$, $S_{back}$

\end{algorithmic}
\end{algorithm}




