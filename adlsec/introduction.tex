\section{Introduction}

Modeling activity of daily life (ADL) has become a burgeoning research topic 
since people demand a comfortable life lifestyle at home
at a lower cost. 
Since HVAC consumes $\sim$53\% of the total electrical usage 
by heating and cooling spaces%\cite{energydatabook2011} 
of an average household, 
automating the operation HVAC devices to save energy is important. 
One of the crucial components required to achieve this goal is 
to model and predict the occupancy of a home. 
Supervised learning approaches on the analysis of indoor temperature~\cite{kleiminger2014predicting}, 
smart phone's GPS data~\cite{koehler2013therml}, 
electricity consumption~\cite{erickson2010occupancy} and 
sensor data by tracking the indoor activities~\cite{scott2011preheat,alrazgan2011learning} 
are effective ways to approach this prediction problem. 
Prediction with sensor data is broadly researched. 
By capturing daily activities like room occupancy of the house, 
usage of electrical devices, 
usage of water system, etc. using sensors, 
researchers have modeled occupancy~\cite{mahmoud2013behavioural,erickson2010occupancy,beltran2014optimal} 
and used these results to automate the control of HVAC system. 

Although the supervised learning kNN~\cite{scott2011preheat}, 
neural network~\cite{mahmoud2013behavioural} and Markov model~\cite{erickson2010occupancy} 
are effective, 
the detailed household activities represented as a time series is not fully utilized. 
Daily activities such as waking up, cooking, washing, commuting to work/school and back, etc. 
have different patterns based on the day. 
For instance, the schedule on a working day is significantly different from that of a weekend 
or a holiday. 
Thus, this scenario leads itself to an episode mining analysis, 
which can be used to predict household occupancy. 
Following this strategy of episode mining for occupancy prediction has three advantages. 
First, episode mining, a temporal mining approach, mines according to the time 
distribution for each type of activity. 
Second, it builds the activity scenario and connects the episode with 
a probabilistic hidden Markov model (HMM). 
As opposed to previous models, 
the time and order of each kind of activity are fully utilized. 
Third, the algorithm predicts according to the scenarios-based probabilistic model episode 
generative HMM (EGH). The prediction accuracy improves compares to the existing models. 

Our contributions can be highlighted in the form of the three questions below. 
\begin{enumerate}
\item How can we mine for meaningful scenarios? 
Episode mining can mine many frequent episodes, but not all the episodes are useful 
for occupancy prediction. 
By narrowing the episodes according to the start state, 
end time, event dwelling time and gap between two activities, 
we can interpret these episodes and provide insight as to 
which episodes are informative. 
\item How can we predict the occupancy more accurately?
Our dataset comprises of detailed information of the various activities of a household 
tracked as a time series on a daily basis. 
Thus our episodes have rich detailed information based on occupancy and 
un-occupancy of the household. 
Since we are mining episodes from this data, 
the accuracy of occupancy prediction improves significantly. 
\item  Can it help save electric usage at home?
The prediction occurs at least 15-minutes ahead of a person 
leaving or coming back. 
By connecting this prediction result to an automatic controlling system 
over HVAC, the HVAC can be operated ahead. 
Since the HVAC does not work during occupancy, 
it saves electric usage. 
\end{enumerate}

\iffalse
Next, 
we first discuss the time-gap constraint episode mining 
model and the mixture model, 
and how to predict the target event in section 4. 
Then, 
in section 5 we will show 
the experimental results. 
\fi

%Other similar research field is the mobility of person, from work to home. 
%\cite{baumann2013influence} is similar to our work but from work  to home. It utilizes the second-order Markov Model. MAJOR is the novel approach. 

%\cite{baumann2013long} gives a good definition on the prediction. It is very related to our work. 

%\cite{kleiminger2013inferring} has a home set algorithm related to time/day. It has a beautiful figure on the household occupancy. 






