\section{Conclusion}
Residential occupancy prediction is a hot research topic on controlling the HVAC. 
The accuracy of occupancy prediction influences the comfortability of persons inside 
the home and energy saving. 
In order to achieve the highest prediction result, 
we propose to integrate the mixture EGH model and 
kNN together as a hydrate approach.

Our work differs from previous research based on the main contributions listed below:
\begin{enumerate}
\item We formulate the problem as one of temporal mining: the activities inside the building are abstracted as episodes, and each episode is connected with an episode generative HMM model.
\item We mine the activity patterns according to the time and gap: both the duration of each type of 
activity, and the gap between two consecutive events are limited in a proper range. 
This range is extracted from the historical data according to the weekday and holidays.
\item Our hydrate prediction solution performs best on the workday occupancy prediction: 
in case of normal activities, we apply mixture EGH model; 
in case of abnormal events, we utilize kNN,  
which is generally considered a benchmark in occupancy prediction problem. 
\end{enumerate}
%In this paper, we propose the mixture EGH model and compare it with two other 
%benchmark models, probability density function and kNN approach. 
%The results show that it generally performs better than kNN to predict the 
%occupancy and un-occupancy states in the workdays. 
%The mixture model predicts well for the period of after person getting up and before person 
%going out. 
%The coefficient of the episode generative HMM models helps 
%predict the exact leaving time. 
%However in the case of abnormal events, 
%kNN performs good because it can average the 
%historical data. 
%Even if there is an abnormal day, 
%kNN can leverage it. 

In the future work, we will continue working on the holiday occupancy prediction. 
The occupancy patterns for these days are completely different. 
For example, in certain weekdays, a person may never goes out. 
Therefore the occupancy prediction probably depend more on date other than the indoor activities. 
Further, we will apply this temporal mining approach on the GPS datasets~\cite{koehler2013therml}
to check the effectiveness of occupancy prediction with different kinds of data. 