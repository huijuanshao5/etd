\section{Related Work}
Most of the approaches that model and predict occupancy primarily use sensor data 
such as room occupancy, use of electrical appliances, water usage, etc.
Several supervised learning approaches like kNN, neural networks, rule-based, 
and Markov chain models have been used to model and predict building occupancy 
\cite{scott2011preheat, alrazgan2011learning, mahmoud2010occupancy, mahmoud2013behavioural, erickson2014occupancy, beltran2014optimal}.  
Using the kNN supervised learning algorithm and monitoring sensor data 
for portion of the day, 
\cite{scott2011preheat} predicts the entire day's occupancy. 
A neural network approach using binary time series based on 
occupancy/unoccupancy along with exogenous input network (NARX) was 
proposed in \cite{mahmoud2010occupancy, mahmoud2013behavioural}. 
Mahmoud et al. tackle the problem by presenting a non-linear autoregressive 
with exogenous input (NARX) network. 
Several Markov chain models like blended Markov chain, 
closest distance Markov chain, 
and moving-window Markov chains were presented in \cite{erickson2014occupancy}. 
A mixture of multi-lag Markov chains was used to predict occupancy of 
single person offices \cite{manna2013learning}. 
In that work, the authors also compare their model with Input Output Hidden Markov Model, 
First Order Markov Chain and the NARX neural network. 

Our work differs from previous research based on the main contributions listed below:
\begin{enumerate}
\item We formulate the problem as one of temporal mining: the activities inside the building are abstracted as episodes, and each episode is connected with an episode generative HMM model.
\item We mine the activity patterns according to the time and gap: both the duration of each type of 
activity, and the gap between two consecutive events are limited in a proper range. 
This range is extracted from the historical data according to the weekday and holidays.
\item Our prediction solution performs better than the kNN approach mentioned before, 
which is generally considered a benchmark in occupancy prediction problem. 
\end{enumerate}

%%%%%comment several paragraphs
\iffalse
These superseded learning approaches are classified into several categories. 
The first is on the probability density distribution of key events. 
\cite{tominaga2012unified} proposes that at a time,  a person goes out has a Bernoulli distribution. 
The second effective benchmark approach is kNN. 
kNN approach is employed in 
\cite{scott2011preheat} to predict the occupancy of the left day 
after knowing the occupancy in the partial day. 
It splits the whole day's time into 96 15-minutes intervals 
then to find the top-5 similar day in the training date. 
The average of these similarity is the predictive occupancy. 
The third is the pattern discovery by rule and neural network. 
A rule-based approach is proposed by
\cite{alrazgan2011learning}  for occupancy prediction under the frame work of Decision Guidance Query Language (DGQL). 
A variant of neural network has been proposed by 
\cite{mahmoud2010occupancy}. \cite{mahmoud2010occupancy} converts the data into binary occupancy/unoccupancy data in the first step. Then a model name non-linear autoregressive network with exogenous input (NARX) network is modeled for prediction. 
\cite{mahmoud2013behavioural} also uses binary time series with NARX network. 
The last are models related to Markov chains. 
Several Markov Chains have been compared in the paper of \cite{erickson2014occupancy}, including blended Markov Chain, closest distance Markov Chain, and the moving window Markov Chain with respect to modeling occupancy. 
\cite{erickson2010occupancy} uses moving-window markov chain for occupancy prediction. 
\cite{erickson2013poem} utilizes the markov chain model and blend markov chain model for prediction. 
\cite{beltran2014optimal} uses a blend-Markov chain model for prediction. 
\cite{manna2013learning} uses mixture of multi-lag Markov chains to predict the occupancy in single person offices. It compares with other previous approaches Input Output Hidden Markov Model, First order Markov Chain and NARX Neural Network. 

This work contributes the follows:
1) formulate the problem as a temporal mining problem;
2) mine the activity patterns according to time and gap;
3) the occupancy prediction performance of this temporal mining approach works better than kNN for most cases.
\fi
