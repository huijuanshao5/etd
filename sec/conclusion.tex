\chapter{Conclusion}

Smart buildings research involves many topics. 
We focus on two of them energy/water disaggregation and occupancy prediction. 
In this work, we utilize temporal mining algorithms 
frequent episode mining to discover the usage patterns 
of electrical devices and water use ends. 
Based on these frequent episodes, 
we can disaggregate individual devices from the multi-phases 
or single-phase aggregated data. 

Also, we mine the frequent episodes from the room position of a 
person in a residential building. 
Then we connect these frequent episodes with HMM 
to build Episode Generating HMM for target event un-occupancy prediction. 

In the future work, 
regarding energy disaggregation, we will incorporate more feature 
when a device starts up and improve this temporal mining 
approach to disaggregate more devices. 
For water disaggregation, since there's limitation in 
discovering specific water use ends, 
we will explore temporal mining approach, 
and integrate the dynamic time warping with motif mining. 

As to occupancy prediction, 
the hybrid approach by integrating kNN and a mixture of EGH 
performs best on sensor data set. 
Next step, we will apply this approach to GPS dataset 
on occupancy prediction. 

\subsection{Survey Conclusion}
Significant increase in energy usage worldwide and the consequent impact
on the environment has pushed energy disaggregation research to the
forefront in recent years. 
While energy disaggregation primarily refers to electricity disaggregation,
similar algorithms are being explored for natural gas and water disaggregation. 
%Majority of algorithms used in energy disaggregation
%can also be applied to gas, water disaggregation.
We have surveyed features, algorithms, evaluation metrics, and 
instrumentation required for energy disaggregation 
from the perspective of data mining. 
Initially, disaggregation algorithms focused on features
of real power and reactive power, which 
can be easily obtained from low frequency data.
With decreasing cost of meters to record data,
high frequency consumption data can be recorded these days. 
Therefore, rich features such as harmonics, transient shapes, 
noise data, and electromagnetic fields are available which increase accuracy. 
While supervised algorithms were first used in energy disaggregation, 
it is becoming more common to use unsupervised algorithms.
Although there is no unified evaluation metric for energy disaggregation 
so far, there are two types: 
event-based and time-series based. An important need for the research community is to 
agree on a standardized evaluation metric. This will assist researchers in
comparing and improving
their algorithms' performance.
In addition, we describe the setup of experiments on 
how to record data.  
In the near future, 
more data mining algorithms will be designed and invented in the 
energy disaggregation area, thus improving the disaggregation accuracy and
scalability, and enabling its widespread use. 

\subsection{Energy Disaggregation Conclusion}
We have described an intuitive motif-based approach to disaggregation that performs
well relative to more complex algorithms that perform detailed modeling of temporal
profiles.
More importantly,
we have demonstrated how our approach is not just an aid to disaggregation
but, as a byproduct, also extracts temporal episodic relationships that shed
insight into consumption patterns. In this sense, our work goes further
than past work into addressing the real goal of disaggregation research,
namely, to understand systematic trends in consumption patterns with a view
toward identifying opportunities for savings.

\subsection{Multivariate Energy Disaggregation Conclusion}
This chapter proposes an semi-supervised recursive multivariate piecewise motif mining approach 
to electricity disaggregation. 
%Before unsupervised disaggregation, 
We use a period of time's data to find the features of individual devices, 
including the power level and standard deviation.
Based on these features, 
we recursively utilize multivariate motif mining to uncover devices 
which draw power from both phases equally or unequally, 
then separate devices from each phase with motif mining. 
Large power devices are removed from two phases in the first a few rounds, 
which brings the benefit of decreasing the noise caused by large power devices 
during the process of disaggregating from single phase. 
Therefore more devices with small power consumption are separated 
in the piecewise motif mining from single phase. 
Also, this piecewise motif mining approach can identify
continuously variable loads such as heatingOutdoor. 
Besides, we apply motif mining approach to water disaggregation, 
it can separate water use ends which has a steady water usage state such as shower device, 
but cannot disaggregate those water use ends which consumes water variable for the whole 
cycle like toilet. 

In the future, 
on energy disaggregation we will use more features from the multiple phases' aggregated data, 
such as startup shape of individual devices.
For water disaggregation, we will explore on how to integrate the 
dynamic time warping with multivariate piecewise motif mining. 

\subsection{Occupancy Prediction Conclusion}
Residential occupancy prediction is a hot research topic on controlling the HVAC. 
The accuracy of occupancy prediction influences the comfortability of persons inside 
the home and energy saving. 
In order to achieve the highest prediction result, 
we propose to integrate the mixture EGH model and 
kNN together as a hydrate approach.

Our work differs from previous research based on the main contributions listed below:
\begin{enumerate}
\item We formulate the problem as one of temporal mining: the activities inside the building are abstracted as episodes, and each episode is connected with an episode generative HMM model.
\item We mine the activity patterns according to the time and gap: both the duration of each type of 
activity, and the gap between two consecutive events are limited in a proper range. 
This range is extracted from the historical data according to the weekday and holidays.
\item Our hydrate prediction solution performs best on the workday occupancy prediction: 
in case of normal activities, we apply mixture EGH model; 
in case of abnormal events, we utilize kNN,  
which is generally considered a benchmark in occupancy prediction problem. 
\end{enumerate}
%In this paper, we propose the mixture EGH model and compare it with two other 
%benchmark models, probability density function and kNN approach. 
%The results show that it generally performs better than kNN to predict the 
%occupancy and un-occupancy states in the workdays. 
%The mixture model predicts well for the period of after person getting up and before person 
%going out. 
%The coefficient of the episode generative HMM models helps 
%predict the exact leaving time. 
%However in the case of abnormal events, 
%kNN performs good because it can average the 
%historical data. 
%Even if there is an abnormal day, 
%kNN can leverage it. 

In the future work, we will continue working on the holiday occupancy prediction. 
The occupancy patterns for these days are completely different. 
For example, in certain weekdays, a person may never goes out. 
Therefore the occupancy prediction probably depend more on date other than the indoor activities. 
Further, we will apply this temporal mining approach on the GPS datasets~\cite{koehler2013therml}
to check the effectiveness of occupancy prediction with different kinds of data. 