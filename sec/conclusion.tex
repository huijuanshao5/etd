\chapter{Conclusion}

In this day and age of big data and big compute, data science has become an essential lens through which we look at the world. Technologies like Internet of Things and the availability of data such as traffic flow, power consumption, air quality, social networks, etc. have made \emph{urban computing} a burgeoning research topic. One of the critical components of \emph{urban computing} research is the analysis of smart buildings. Particularly, energy disaggregation and distribution are important topics because of its two-fold impact on society. Firstly it is an essential resource, which has a great impact on our economy so effective distribution is important. Secondly, the larger impact the energy industry has on the environment, like climate change, makes it an extremely sensitive and relevant topic of research. Thus, in this work, we focus on two important energy related problems in a smart building setting viz. resource disaggregation and occupancy prediction. 

We demonstrate that with the use of frequent episode mining in conjunction with temporal mining techniques, we can effectively glean insights into usage patterns of electrical devices. Our approach describes a novel motif discovery approach that utilizes the on/off events to unravel operation frequency and duration of devices. We show that the our approach is very adroit at discerning multiple power levels and at effectively untwine the combinatorial operation of the devices. Moreover, we also show how our approach is not just an aid to disaggregation but, as a byproduct, also extracts temporal episodic relationships that shed insight into consumption patterns.

To improve upon our initial work, we proposed a semi-supervised recursive multivariate piecewise motif mining approach. Since the algorithm operates in two-phases and effectively filters out appliances that have large power consumption in the first phase in the first phase, it can effectively discover usage patterns of smaller appliances. This insight, provided by out approach, allows for more precise energy disaggregation. Moreover this approach can be effectively utilized to identify continuously variable loads like outdoor heating. 

Solving energy disaggregation is an important practical problem which results in a highly monetizable insight - consumption patterns. As a result policy makers and energy distributors can design packages for consumers based on their needs. This will enable both the consumers and the distributors to effectively and \emph{smartly} buy/sell power, which is highly customized to the needs of a home.

The second aspect of this thesis tackles occupancy prediction. This problem's larger impact on the energy industry primarily involves the automated turning on/off of the HVAC, which is one of the biggest energy consuming appliances. Moreover, occupancy prediction can assist in several IoT applications such as bandwidth allocation of wireless signals based on locality and energy saving applications.
In order to achieve the highest prediction result, 
we propose to integrate the mixture EGH model and 
kNN together as a hydrate approach.

Our work differs from previous research based on the main contributions listed below:
\begin{enumerate}
\item We formulate the problem as one of temporal mining: the activities inside the building are abstracted as episodes, and each episode is connected with an episode generative HMM model.
\item We mine the activity patterns according to the time and gap: both the duration of each type of 
activity, and the gap between two consecutive events are limited in a proper range. 
This range is extracted from the historical data according to the weekday and holidays.
\item Our hydrate prediction solution performs best on the workday occupancy prediction: 
in case of normal activities, we apply mixture EGH model; 
in case of abnormal events, we utilize kNN,  
which is generally considered a benchmark in occupancy prediction problem. 
\end{enumerate}

\textbf{Please put some more insights about the results of occupancy prediction}

\section{Future Work}
While we have made significant headway in energy disaggregation, there is room for significant improvement. One of the immediate extensions is to incorporate more features (in the multi-phased aggregated) when a device turns on. The sudden spikes in the aggregated data when normalized can indicate the \emph{startup shape} which can be corresponded to a device and thus improve the overall effectiveness of device prediction, thus leading to more accurate energy disaggregation. Moreover we can significantly improve the temporal mining approach to disaggregate more devices. Furthermore, our disaggregation algorithms can be explored for water disaggregation as well. We will exploit our temporal mining algorithms integrated with dynamic time warping and motif mining to propose an algorithm to effectively conduct home level disaggregation. We can also extend our disaggregation algorithms for bandwidth distribution for internet service provides. Using our disaggregation algorithms, we can decipher the device level internet usage and plan for effective distribution to a home/neighborhood. 

The occupancy prediction work lends itself to future extensions via hybrid approaches. We can integrate kNN and a mixture of EGH which has the best performance on the sensor data set. One of the future directions is to incorporate GPS based information to track movements of the house residents. This has the potential to be an excellent surrogate to automated power control of devices in a home. Another interesting problem to tackle is holiday occupancy prediction. The occupancy patterns for these days are completely different. For example, in certain weekdays, a person may never goes out. Therefore the occupancy prediction probably depend more on date other than the indoor activities. 

\textbf{put some final remarks here.}
