\chapter{Proposal: Indoor Activities Tracking}
\markright{Huijuan Shao \hfill Chapter 5. Proposal: Indoor Activities Tracking \hfill}

\section{Indoor Activities Tracking}
In the previous work, 
episode mining was applied in energy disaggregation and 
time-gap constrained episode mining and a corresponding EGH model is utilized 
to predict occupancy. 
Episode mining and its extension can be widely used in the area of
sustainability for mining patterns and special prediction target.  

%Motivation 
Non-invasive indoor activities tracking is an interesting topic. 
It's a basis for smart home applications, 
such as monitoring the patient or elderly people, 
controlling the electrical devices based on the room occupancy. 
Most of such tracking system reply on installed cameras or microphones, 
or carried powered devices, such as ultrasonic RF transceiver. 
These devices are considered by some people to be invasive on the privacy. 
Non-invasive resident identification and indoor activity tracking was 
studied in the papers of\cite{srinivasan2010using, hnat2012doorjamb}. 
In the paper of \cite{srinivasan2010using}, 
the authors
install a ultrasonic distance sensor on the 
beam of the door to capture the height of a person 
when he/she walking through the door. 
According to the difference of height, 
each person can be identified at which room 
when staying at home. 
Based on it, \cite{hnat2012doorjamb} installs 
ultrasonic sensors above each doorway to 
estimate which room the person will go. 
The author utilizes the height of a person and walking direction. 
The challenges lies in two problems. 
The person identification is based on the height. 
Therefore the height difference among people at home 
is supposed to be greater than 7cm. 
The walking direction estimation based on the gait is 
not so accurate because there are maybe 
multiple paths instead of one. 

So far, the research on noninvasive indoor activities tracking just begins. 
There are still a lot of questions that need to be answered.
First is on how to estimate the room occupancy of each person at home more effectively?
Second, how to improve the room occupancy prediction accuracy?
Third, how to model a probabilistic time series model for prediction?

\section{Approach}
To solve the above indoor activities tracking problem, 
we propose to use the episode mining approach 
for the room location prediction. 
Each person inside the home is identified by one feature height. 
And he/she generates a time series with 
walking direction and doorway ID. 
Then we can apply the time or gap episode mining 
by constraints on the walking direction events. 
The order of mined episodes can be defined by us. 
For instance, if the door is closed. 
The door should be opened before a person goes through it. 
After that, the episode mining approach is applied to mine the frequent episodes. 
These frequent patterns with time give us the training data 
for target event prediction. 

Unlike the occupancy prediction problem, 
there are several target events instead of one. 
We apply the mixture episode generative HMM model and 
calculate the probability of staying in each room. 
Since a person can only be in one room at home, 
we can select the room with the highest probability as the result. 

We can briefly describe the problem as below. 
Each doorway has an ID. On it, an ultrasonic distance sensor is installed. 
Therefore for each doorway over a period of time, 
a time series data on height measurement and walking direction 
is recorded.  
The events in the doorjamb are of four types
$e_i \in E=\{t_i, d_i, h_i, v_i\}$, where 
$t_i$ is the timestamp, 
$d_i$ is the doorway ID, 
$h_j$ is the height measurement and 
$v_i$ is the walking direction at the time of $t$. 
The target is to estimate the room occupancy of each person inside the home 
over a period of time 
$s_i \in S = (r1_i, r2_i)$ where $r1_i$ denotes that person 1 is at the room location at time $i$. 

\section{Evaluation}
We already know the ground truth,  which is when a person stays in each room in the building. 
We can compare the results of our approach and other approaches 
presented in \cite{hnat2012doorjamb} with the ground truth data. 
After calculating the precision, recall, and f-measure values of these two approaches, 
we can judge which method performs better. 


\iffalse
In the first step, 
we can define the discrete event types. 
There are totally three types of events as below. 
\begin{enumerate}
\item When a person walks through a doorway using data from the 
?doorjamb? sensor. 
\item When a person performs a particular hand gesture using data from 
a smart watch.
\item When an appliance turns on using power meter data.
\end{enumerate}
\fi


