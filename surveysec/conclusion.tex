\section{Conclusion}
\label{sec:conclusion}
Significant increases in energy usage worldwide and the consequent impact
on the environment have pushed energy disaggregation research to the
forefront in recent years. 
%Majority of algorithms used in energy disaggregation
%can also be applied to gas, water disaggregation.
We have surveyed features, algorithms, evaluation measures, and 
instrumentation required for energy disaggregation 
from a data miner's perspective.
While initial
disaggregation algorithms were focused on low-frequency data,
today's installations support
high-frequency data recording.
Therefore, rich features such as harmonics, transient shapes, 
noise characteristics, and electromagnetic field
measurements are available to improve disaggregation performance.
While supervised algorithms were first used in energy disaggregation, 
it is becoming more common to use unsupervised algorithms.
We have also identified the need for the research community to coalesce
around accepted standards of evaluation. 

In this survey, discussions of 
energy disaggregation primarily refer to electricity disaggregation,
but similar algorithms are being explored for natural gas and water disaggregation. 
An interesting area of potential research is to jointly disaggregate multiple
utilities, e.g., electricity and water, which will undoubtedly provide additional
contextual features for analysis.
As smart homes and Internet of Things (IoT) installations become more
pervasive, it is clear that disaggregation research will continue to be
relevant into the foreseeable future.

