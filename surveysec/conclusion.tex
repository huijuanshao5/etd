\section{Conclusion}
\label{sec:conclusion}
Significant increase in energy usage worldwide and the consequent impact
on the environment has pushed energy disaggregation research to the
forefront in recent years. 
While energy disaggregation primarily refers to electricity disaggregation,
similar algorithms are being explored for natural gas and water disaggregation. 
%Majority of algorithms used in energy disaggregation
%can also be applied to gas, water disaggregation.
We have surveyed features, algorithms, evaluation metrics, and 
instrumentation required for energy disaggregation 
from the perspective of data mining. 
Initially, disaggregation algorithms focused on features
of real power and reactive power, which 
can be easily obtained from low frequency data.
With decreasing cost of meters to record data,
high frequency consumption data can be recorded these days. 
Therefore, rich features such as harmonics, transient shapes, 
noise data, and electromagnetic fields are available which increase accuracy. 
While supervised algorithms were first used in energy disaggregation, 
it is becoming more common to use unsupervised algorithms.
Although there is no unified evaluation metric for energy disaggregation 
so far, there are two types: 
event-based and time-series based. An important need for the research community is to 
agree on a standardized evaluation metric. This will assist researchers in
comparing and improving
their algorithms' performance.
In addition, we describe the setup of experiments on 
how to record data.  
In the near future, 
more data mining algorithms will be designed and invented in the 
energy disaggregation area, thus improving the disaggregation accuracy and
scalability, and enabling its widespread use. 
