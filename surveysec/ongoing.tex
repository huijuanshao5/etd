\section{Ongoing Research}
\label{sec:ongoing}
Current analytics
research in energy disaggregation is primarily focused on two areas: 
feature discovery and new learning algorithms.
Feature discovery is primarily driven by
specialists in electrical engineering who bring a nuanced
understanding of device and electricity features to bear upon this
problem. With greater emphasis in the data mining community on deep
learning and other methods that emphasize novel representational
layers, there is considerable scope for data scientists to play
a role here, as research is still in its nascent stages.
In the area of 
learning algorithms, unsupervised methods have a distinct advantage, since
labelled data is not required and today's state-of-the-art
unsupervised methods are quite competitive when compared with supervised
methods. 
%Maybe the number of devices or device types can be assumed, 
%but generally unsupervised disaggregation will become 
%the most widely used in the near future as it relaxes the assumptions made by supervised
%learning techniques.
Research into novel machine learning models will continue to
improve disaggregation performance.

As more companies participate in this area,
new software tools are concomitantly
being developed to help users analyze power consumption data.
Smart!\cite{barker2012smart} provides an interface 
for users to monitor their power consumption. 
A database has been built from the REDD dataset~\cite{lai2012database}.
These tools benefit both developers of new methods and the end consumers.

Non-intrusive load monitoring paves the way for many other research problems,
which are not surveyed here.
One such area
is occupancy research, which infers whether there are people occupying a
building within particular segments of space or time~\cite{chen2013non}.
The second area is demand response~\cite{albadi2007demand}, 
which refers to ways by which utilities balance supply and demand by 
offering incentives to consumers to reduce or shift their peak energy
demand.
Inferring activities of daily life~\cite{song2014short}, 
promoting personal energy savings~\cite{lee2014personalized}, and
efficient energy management~\cite{collins2012smart} are all current
topics of research.

