\section{Ongoing Research}
\label{sec:ongoing}
Current data mining research in energy disaggregation focuses on two areas: 
feature discovery and developing learning algorithms for disaggregation.
Feature discovery is mainly explored by electrical engineering specialists 
who have a better understanding of device or electricity features.
There is a lot of scope for data scientists to extend the research in developing
learning techniques as this part of research is still in its nascent stages. 
For learning algorithms, unsupervised ones have a distinct advantage since
labeled data is not required. 
%Maybe the number of devices or device types can be assumed, 
%but generally unsupervised disaggregation will become 
%the most widely used in the near future as it relaxes the assumptions made by supervised
%learning techniques.
The introduction of more new statistical models 
will improve the disaggregation accuracy.

As more electric companies join this field, 
software tools have been developed to analyze data on power consumption. 
Smart!\cite{barker2012smart} provides an interface tool 
so that users can monitor their power consumption. 
A database has been built for REDD~\cite{lai2012database}.
These tools benefit both the developers and customers. 

Non-intrusive load monitoring paves the way for many other research problems. 
One of them is occupancy research, which infers whether there are people living in the home~\cite{chen2013non}.
The second one is demand response. Inferring activities of daily life~\cite{song2014short}, 
personal energy usage~\cite{lee2014personalized}, efficient energy management~\cite{collins2012smart} are all topics of research.




